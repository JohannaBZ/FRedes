\usepackage[spanish]{babel}
\usepackage[utf8]{inputenc}
\usepackage{graphicx}
\usepackage{listings}             % Incluye el paquete listings
\usepackage[T1]{fontenc}
\usepackage{color}
\usepackage{subcaption}
\usepackage{caption}
%\usepackage[backend=bibtex]{biblatex}
%\addbibresource{MiBiblio.bib}
\usepackage{hyperref} %para las refrencias
    \hypersetup{breaklinks=true,colorlinks=true,
        linkcolor=black,citecolor=black,urlcolor=black}
\usepackage{natbib}
\usepackage{xcolor}
\definecolor{miverde}{rgb}{0,0.6,0}
\definecolor{migris}{rgb}{0.5,0.5,0.5}
\definecolor{mimalva}{rgb}{0.58,0,0.82}
\definecolor{gray97}{gray}{.97}
\definecolor{gray75}{gray}{.75}
\definecolor{gray45}{gray}{.45}


\lstset{ %
  backgroundcolor=\color{gray97},   % Indica el color de fondo; necesita que se a�ada \usepackage{color} o \usepackage{xcolor}
  basicstyle=\footnotesize,        % Fija el tama�o del tipo de letra utilizado para el c�digo
  breakatwhitespace=false,         % Activarlo para que los saltos autom�ticos solo se apliquen en los espacios en blanco
  breaklines=true,                 % Activa el salto de l�nea autom�tico
  captionpos=b,                    % Establece la posici�n de la leyenda del cuadro de c�digo
  commentstyle=\color{miverde},    % Estilo de los comentarios
  deletekeywords={...},            % Si se quiere eliminar palabras clave del lenguaje
  escapeinside={\%*}{*)},          % Si quieres incorporar LaTeX dentro del propio c�digo
  extendedchars=true,              % Permite utilizar caracteres extendidos no-ASCII; solo funciona para codificaciones de 8-bits; para UTF-8 no funciona. En xelatex necesita estar a true para que funcione.
  %frame=single,	                   % A�ade un marco al c�digo
  keepspaces=true,                 % Mantiene los espacios en el texto. Es �til para mantener la indentaci�n del c�digo(puede necesitar columns=flexible).
  keywordstyle=\color{blue},       % estilo de las palabras clave
  language=Pascal,                 % El lenguaje del c�digo
  otherkeywords={*,...},           % Si se quieren a�adir otras palabras clave al lenguaje
  numbers=left,                    % Posici�n de los n�meros de l�nea (none, left, right).
  numbersep=5pt,                   % Distancia de los n�meros de l�nea al c�digo
  numberstyle=\small\color{migris}, % Estilo para los n�meros de l�nea
  rulecolor=\color{black},         % Si no se activa, el color del marco puede cambiar en los saltos de l�nea entre textos que sea de otro color, por ejemplo, los comentarios, que est�n en verde en este ejemplo
  showspaces=true,                % Si se activa, muestra los espacios con guiones bajos; sustituye a 'showstringspaces'
  showstringspaces=false,          % subraya solamente los espacios que est�n en una cadena de esto
  showtabs=true,                  % muestra las tabulaciones que existan en cadenas de texto con gui�n bajo
  stepnumber=2,                    % Muestra solamente los n�meros de l�nea que corresponden a cada salto. En este caso: 1,3,5,...
  stringstyle=\color{mimalva},     % Estilo de las cadenas de texto
  tabsize=2,	                   % Establece el salto de las tabulaciones a 2 espacios
  %title=\lstname                   % muestra el nombre de los ficheros incluidos al utilizar \lstinputlisting; tambi�n se puede utilizar en el par�metro caption
}
